%%%%%%%%%%%%%%%%%%%%%%%%%%%%%%%%%%%%%%%%%
% Thin Sectioned Essay
% LaTeX Template
% Version 1.0 (3/8/13)
%
% This template has been downloaded from:
% http://www.LaTeXTemplates.com
%
% Original Author:
% Nicolas Diaz (nsdiaz@uc.cl) with extensive modifications by:
% Vel (vel@latextemplates.com)
%
% License:
% CC BY-NC-SA 3.0 (http://creativecommons.org/licenses/by-nc-sa/3.0/)
%
%%%%%%%%%%%%%%%%%%%%%%%%%%%%%%%%%%%%%%%%%

%----------------------------------------------------------------------------------------
%	PACKAGES AND OTHER DOCUMENT CONFIGURATIONS
%----------------------------------------------------------------------------------------

\documentclass[a4paper, 11pt]{article} % Font size (can be 10pt, 11pt or 12pt) and paper size (remove a4paper for US letter paper)

\usepackage[protrusion=true,expansion=true]{microtype} % Better typography
\usepackage{graphicx} % Required for including pictures
\usepackage{wrapfig} % Allows in-line images
\usepackage{listings}

\usepackage{mathpazo} % Use the Palatino font
\usepackage[T1]{fontenc} % Required for accented characters
\linespread{1.05} % Change line spacing here, Palatino benefits from a slight increase by default

\makeatletter
\renewcommand\@biblabel[1]{\textbf{#1.}} % Change the square brackets for each bibliography item from '[1]' to '1.'
\renewcommand{\@listI}{\itemsep=0pt} % Reduce the space between items in the itemize and enumerate environments and the bibliography

\renewcommand{\maketitle}{ % Customize the title - do not edit title and author name here, see the TITLE block below
\begin{center} % Right align
{\LARGE\@title} % Increase the font size of the title

\vspace{10pt} % Some vertical space between the title and author name

{\large\@author} % Author name
\\\@date % Date

\vspace{10pt} % Some vertical space between the author block and abstract
\end{center}
}

%----------------------------------------------------------------------------------------
%	TITLE
%----------------------------------------------------------------------------------------

\title{\textbf{Buffer Overflow} Code Explanation % Title
} % Subtitle

\author{\textsc{Bimal Varghese} % Author
\\{\textit{FISAT}}} % Institution

\date{\today} % Date

%----------------------------------------------------------------------------------------

\begin{document}

\maketitle % Print the title section



%----------------------------------------------------------------------------------------
%	ESSAY BODY
%----------------------------------------------------------------------------------------

\section*{Sample Code}

\begin{lstlisting}
#include<stdio.h>
#include<string.h>
#include<stdlib.h>

int main(inr argc,char* argv[])
{
	char buf[400]
	
	if (argc <2)
	{
		printf("Please enter command line arguments\n");
		exit(0);
	}
	
	strcpy(buf,argv[1]);
	return 0;
}

\end{lstlisting}

%------------------------------------------------

\section*{Explanation}
The \textit{strcpy()} function does not check whether or not the number of bytes being copied to the buffer is less than the size of the buffer.The \textit{strcpy()} function copies  the command line argument passed to the program directly, without checking the number of bytes. As a result,if the user give an input string of length greater than 400(a little more than 400 ), the buffer over flows and the memory address adjacent to buffer, \textbf{ebp} and \textbf{eip}, gets overwritten.

\begin{verbatim}
gcc example.c -o example
./example 'python -c 'print "A" *412'
\end{verbatim}

	When the above program executed with given command line arguments, the \textbf{ebp} register get overflowed with 0x41, which is the hex value of '\textbf{A}'.

\section*{Overflow Attack}	
	To perform overflow attack, we need to over flow the \textbf{eip} register, because \textbf{eip} points to the next instruction. \textbf{eip} lies next to the \textbf{ebp} register, and hence is just 4 bytes away. So if we run 
\begin{verbatim}
gcc example.c -o example
./example 'python -c 'print "A" *416'
\end{verbatim}
	the \textbf{eip} pointer gets over written.
	
	Now instead of giving a single string value, if we insert a code which can spawn a shell then we will have a shell where commands can be executed. For that, we have to point the eip to  a assembly code that will spawn a shell (\textit{/bin/sh}).
	
\begin{verbatim}
gcc example.c -o example
./example 'python -c 'print "A" *412 + <shell code>'
\end{verbatim}


	On executing the above program, a shell is spawned with the same privileges as the program. If the program is having a high privilege then attacker will have elevates access rights. If the root user have created the binary, then spawned shell give complete access to the attacker.
%------------------------------------------------









%----------------------------------------------------------------------------------------
%	BIBLIOGRAPHY
%----------------------------------------------------------------------------------------

%\bibliographystyle{unsrt}

%\bibliography{sample}

%----------------------------------------------------------------------------------------

\end{document}